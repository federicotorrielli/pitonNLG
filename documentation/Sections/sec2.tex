\section{Analisi del linguaggio}
\subsection{Intro}
La classe \texttt{PhraseAnalisys.py} è incaricata dell'analisi delle frasi.
Presa in input una frase, il metodo \texttt{correct\_phrase} grazie ad una fuzzy correction rimpiazza e corregge le parole che presentano eventuali errori di battitura, la cosa ci interessa particolarmente poiché non vogliamo perdere parole chiave che saranno utili in seguito.
A partire dalla frase corretta  utilizziamo spacy che, presa in input l'intera frase restituisce l'albero delle dipendenze tra le parole, quindi tutti i tag del part-of-speech e delle dipendenze semantiche, oltre ad altre informazioni utili come la polarità e la presenza di interrogazioni. Sulla base di questo albero analizzeremo vari aspetti delle frasi. L'analisi viene effettuata su ogni frase data in input.
\subsection{Come sfruttiamo l'albero delle dipendenze}
Come si vede dall'immagine \ref{fig:Spacy}, \href{https://spacy.io/}{spacy} costruisce l'albero delle relazioni sintattiche, ogni arco è etichettato con il tipo di relazione che vi è tra le parole.
\begin{figure}[h]
    \centering
    \includegraphics[scale=0.45]{Images/imgSpacy.png}
    \caption{Dependency parsing con Spacy}
    \label{fig:Spacy}
\end{figure}
A partire dall'albero ottenuto con spacy andiamo a ricercare quattro caratteristiche della frase:
\begin{itemize}
    \item Se è una domanda;
    \item Se è utile: cioè se la frase contiene ingredienti (in caso di utilità/non utilità verrà processata diversamente come si vedrà in seguito);
    \item Se contiene le parole yes/no;
    \item La sua polarità, cioè se è una frase affermativa o negativa;
\end{itemize}
\subsection{Frasi interrogative}
Per verificare se la frase sia o meno una domanda usiamo il metodo check\_if\_question in cui semplicemente verifichiamo se è presente un punto interrogativo o se la frase inizia con specifiche parole come "is", "are", "does", "do", "how" ecc...
\subsection{Frasi utili}
La verifica dell'utilità di una frase avviene nel metodo check\_if\_useful, qui percorriamo tutta la frase e analizziamo le relazioni che coinvolgono parole utili. La lista delle parole utili che cerchiamo all'interno di una frase è contenuta nella knowledge\_base che contiene tutti gli ingredienti. Se un ingrediente è formato da più parole come "valerian spring", "mistletoe berry" o  "standard ingredient" analizziamo l'albero delle dipendenze cercando delle specifiche relazioni tra parole per riconoscerli. Per gli ingredienti nelle pozioni analizzate abbiamo riconosciuto quattro tipologie diverse di relazioni tra le parole:
\begin{itemize}
    \item \textbf{amod}: modificatore aggettivale, per ingredienti del tipo \textit{"Standard Ingredient"};
    \item \textbf{dobj}: se l'ingrediente è oggetto diretto del verbo come \textit{"The potion contains chicken"};
    \item \textbf{poss}: relazioni possessive, per ingredienti come \textit{"bicorn's horn"};
    \item \textbf{compound}: nomi composti, per catturare ingredienti come \textit{"lether river water"};
    \item \textbf{nsubj}: per i casi in cui l'ingrediente è il soggetto della frase ma non copre uno dei casi precedenti, quindi ingredienti formati da una sola parola come \textit{"chicken"}, \textit{"spiders"};
    \item \textbf{prep}: quando abbiamo frasi con \textit{"There are ..."};
    \item \textbf{acomp} e \textbf{attr}: per frasi base come \textit{"The last ingredients are cherries"}, molto spesso quando l'ingrediente si trova dopo \textit{is} oppure \textit{are}.
\end{itemize}
Se una frase contiene una o più parole che soddisfano uno di questi casi la frase viene etichettata come \textbf{useful}

\subsection{Frasi yes/no e polarità}
Per alcune tipologie di domande che Piton porrà allo studente sarà richiesta una risposta di tipo yes/no per riconoscere queste frasi il metodo \texttt{check\_yesno} verifica se la frase inizia con yes o con no o se non contiene affatto una di queste parole, si vedrà in seguito come reagirà Piton.
Per la polarità invece analizziamo l'albero delle dipendenze, riconosciamo polarità negativa se:
\begin{itemize}
    \item[i)] l'etichetta morph indica "Polarity=Neg";
    \item[ii)] l'etichetta dep indica "neg";
    \item[iii)] c'è un "no" che precede gli ingredienti per frasi del tipo \textit{"There are no cherries in this potion"}.
\end{itemize}
Diversamente la polarità sarà positiva
\subsection{Esempio di analisi di una frase}
Per chiariare tutto quello che è stato detto si mostra ora un esempio di analisi di una frase.
\begin{lstlisting}
  strin = PhraseAnalisys("There are cheries in this potion")
  pprint(strin.phrase)
  pprint(strin.dependency_tree())
  print(f"Is useful: {strin.check_if_useful()}")
  print(f"Polarity: {strin.check_polarity()}")
  print(f"Is question: {strin.check_if_question()}")
\end{lstlisting}
In questo esempio è stato volutamente commesso un errore di battitura sull'ingrediente cherries per dimostrare il funzionamento della correzione automatica dei refusi citata all'inizio.
\newline
\newline
Output:
\begin{lstlisting}
  'there are cherries in this potion '

  {'are': ('ROOT', 'are'),
   'cherries': ('attr', 'are'),
   'in': ('prep', 'cherries'),
   'potion': ('pobj', 'in'),
   'there': ('expl', 'are'),
   'this': ('det', 'potion')}

  Is useful: True
  Polarity: True
  Is question: False
\end{lstlisting}

Come ci aspettiamo l'analizzatore vede e corregge l'errore, riconosce la presenza dell'ingrediente, che non si tratta di una domanda è che la polarità è positiva.